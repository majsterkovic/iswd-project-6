\documentclass[a4paper,12pt]{article}

\usepackage{graphicx} 
\usepackage[T1]{fontenc}
\usepackage{polski}
\usepackage[utf8]{inputenc}
\usepackage[a4paper]{geometry}
\usepackage{xcolor}
\usepackage{float}

\usepackage{graphicx} 
\usepackage[T1]{fontenc}
\usepackage[polish]{babel}
\usepackage[utf8]{inputenc}
\usepackage{caption}
\usepackage[a4paper]{geometry}
\usepackage{amsmath}
\usepackage[section]{placeins}
\usepackage[tagged, highstructure]{accessibility}
\usepackage{tabularx}
\usepackage{pdfcomment}
\usepackage{xcolor}
\usepackage{float}

\author{\textcolor{blue}{autorzy}}
\date{\textcolor{blue}{data utworzenia raportu}}

\title{Graniczna Analiza Danych - raport}

\begin{document}
\maketitle

\textcolor{blue}{W poszczególnych sekcjach przedstaw wyniki i wnioski z analizy efektywności lotnisk. Oprócz samych tabel mile widziane będą dodatkowe wnioski wyciągnięte z poszczególnych części analizy. \textbf{Wszystkie wyniki powinny być zaokrąglone do 3 miejsc po przecinku}.}

\section{Efektywność}
\textcolor{blue}{W poniższej tabeli przedstaw otrzymane wartości efektywności dla wszystkich lotnisk.}
\begin{table}[H]
    \centering
    \begin{tabular}{c|c}
    \hline
         Lotnisko & Efektywność  \\ \hline
         WAW & 1.000 \\
         KRK & \ldots \\
         \ldots & \ldots \\
         \hline
    \end{tabular}
    \caption{Wartości efektywności dla analizowanych lotnisk}
    \label{tab:airports-efficiency}
\end{table}

\noindent \textcolor{blue}{Wypisz lotniska efektywne oraz nieefektywne:} \\
Lotniska efektywne: \ldots \\
Lotniska nieefektywne: \ldots

\section{Hipotetyczna jednostka porównawcza oraz potrzebne poprawki}

\textcolor{blue}{W tabeli poniżej przedstaw otrzymane wartości wejść hipotetycznej jednostki porównawczej poprawki potrzebne do osiągnięcia efektywności (tylko wejścia) dla wszystkich nieefektywnych lotnisk.}

\begin{table}[H]
    \centering
    \begin{tabular}{c|cccc|cccc}
    \hline
        & \multicolumn{4}{c|}{HCU} & \multicolumn{4}{c}{Poprawki}  \\
         Lotnisko & $i_1$ & $i_2$ & $i_3$ & $i_4$ & $i_1$ & $i_2$ & $i_3$ & $i_4$ \\ \hline
         WAW & \ldots & \ldots & \ldots & \ldots & \ldots & \ldots & \ldots & \ldots \\
         \ldots & \ldots & \ldots & \ldots & \ldots & \ldots & \ldots & \ldots & \ldots \\
    \end{tabular}
    \caption{Wartości wejść hipotetycznej jednostki porównawczej oraz poprawki potrzebne do osiągnięcia efektywności dla nieefektywnych lotnisk }
    \label{tab:airports-hcu-and-improvements}
\end{table}

\section{Superefektywność}

\textcolor{blue}{Otrzymane wartości superefektywności dla \textbf{wszystkich} lotnisk przedstaw w tabeli poniżej.}

\begin{table}[H]
    \centering
    \begin{tabular}{c|c}
    \hline
         Lotnisko & Superefektywność  \\ \hline
         WAW & \ldots \\
         \hline
    \end{tabular}
    \caption{Wartości superefektywności dla analizowanych lotnisk}
    \label{tab:airports-super-efficiency}
\end{table}

\section{Efektywność krzyżowa}

\textcolor{blue}{W poniższej tabeli przedstaw macierz efektywności krzyżowych dla wszystkich lotnisk oraz ich średnie efektywności krzyżowa.}    

\begin{table}[H]
\resizebox{\textwidth}{!}{
\begin{tabular}{c|ccccccccccc|c}
\hline
& WAW & KRK & KAT & WRO & POZ & LCJ & GDN &SZZ & BZG & RZE & IEG & $CR_{avg}$ \\ \hline
WAW & 1.000 & \ldots & \ldots & \ldots & \ldots & \ldots & \ldots & \ldots & \ldots & \ldots & \ldots & \ldots \\
\ldots & \ldots & \ldots & \ldots & \ldots & \ldots & \ldots & \ldots & \ldots & \ldots & \ldots & \ldots & \ldots \\ 
\hline
\end{tabular}}
\label{tab:airports-cross-efficiency}
\end{table}
            
\section{Rozkład efektywności}
\textcolor{blue}{W tej sekcji pokaż wyniki rozkładu efektywności (podział na 5 przedziałów) oraz oszacowaną oczekiwaną wartość efektywności dla wszystkich lotnisk.}
\begin{table}[H]
\begin{tabular}{c|ccccc|c}
\hline
    & $[0-0.2)$ & $[0.2-0.4)$ & $[0.4-0.6)$ & $[0.6-0.8)$ & $[0.8-1.0]$ & $EE$    \\ \hline
WAW & 0.00      & \ldots & \ldots & \ldots & \ldots & \ldots \\
\ldots & \ldots & \ldots & \ldots & \ldots & \ldots & \ldots \\
\hline
\end{tabular}
\label{tab:efficiency-distribution}
\end{table}

\section{Rankingi jednostek}
\textcolor{blue}{Przedstaw i porównaj rankingi lotnisk uzyskane różnymi metodami (superefektywność, średnia efektywność krzyżowa oraz oczekiwana wartość efektywności).}

\noindent Superefektywność: $WAW \succ \ldots \succ \ldots$ \\
Średnia efektywność krzyżowa: \ldots \\
Oczekiwana wartość efektywności: \ldots \\

\textcolor{blue}{Tu przedstaw wnioski z porównania rankingów.}

\end{document}